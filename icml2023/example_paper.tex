%%%%%%%% ICML 2023 EXAMPLE LATEX SUBMISSION FILE %%%%%%%%

\documentclass{article}

% Recommended, but optional, packages for figures and better typesetting:
\usepackage{microtype}
\usepackage{graphicx}
\usepackage{subfigure}
\usepackage{booktabs}  % for professional tables

% hyperref makes hyperlinks in the resulting PDF.
% If your build breaks (sometimes temporarily if a hyperlink spans a page)
% please comment out the following usepackage line and replace
% \usepackage{icml2023} with \usepackage[nohyperref]{icml2023} above.
\usepackage{hyperref}


% Attempt to make hyperref and algorithmic work together better:
\newcommand{\theHalgorithm}{\arabic{algorithm}}

% Use the following line for the initial blind version submitted for review:
%\usepackage{icml2023}

% If accepted, instead use the following line for the camera-ready submission:
\usepackage[accepted]{icml2023}

% For theorems and such
\usepackage{amsmath}
\usepackage{amssymb}
\usepackage{mathtools}
\usepackage{amsthm}

% if you use cleveref..
\usepackage[capitalize,noabbrev]{cleveref}

%%%%%%%%%%%%%%%%%%%%%%%%%%%%%%%%
% THEOREMS
%%%%%%%%%%%%%%%%%%%%%%%%%%%%%%%%
\theoremstyle{plain}
\newtheorem{theorem}{Theorem}[section]
\newtheorem{proposition}[theorem]{Proposition}
\newtheorem{lemma}[theorem]{Lemma}
\newtheorem{corollary}[theorem]{Corollary}
\theoremstyle{definition}
\newtheorem{definition}[theorem]{Definition}
\newtheorem{assumption}[theorem]{Assumption}
\theoremstyle{remark}
\newtheorem{remark}[theorem]{Remark}

% Todonotes is useful during development; simply uncomment the next line
%    and comment out the line below the next line to turn off comments
%\usepackage[disable,textsize=tiny]{todonotes}
\usepackage[textsize=tiny]{todonotes}


% The \icmltitle you define below is probably too long as a header.
% Therefore, a short form for the running title is supplied here:
\icmltitlerunning{}

\begin{document}

\twocolumn[
\icmltitle{Reinforcement Learning}

% It is OKAY to include author information, even for blind
% submissions: the style file will automatically remove it for you
% unless you've provided the [accepted] option to the icml2023
% package.

% List of affiliations: The first argument should be a (short)
% identifier you will use later to specify author affiliations
% Academic affiliations should list Department, University, City, Region, Country
% Industry affiliations should list Company, City, Region, Country

% You can specify symbols, otherwise they are numbered in order.
% Ideally, you should not use this facility. Affiliations will be numbered
% in order of appearance and this is the preferred way.
%\icmlsetsymbol{equal}{*}

\begin{icmlauthorlist}
\icmlauthor{Zhang Fengchen}{}
%
%\icmlauthor{}{sch}
%\icmlauthor{}{sch}
\end{icmlauthorlist}


\icmlcorrespondingauthor{Zhang Fengchen}{fczhang0606@gmail.com}
% You may provide any keywords that you
% find helpful for describing your paper; these are used to populate
% the "keywords" metadata in the PDF but will not be shown in the document
\icmlkeywords{}

\vskip 0.3in
]

% this must go after the closing bracket ] following \twocolumn[ ...

% This command actually creates the footnote in the first column
% listing the affiliations and the copyright notice.
% The command takes one argument, which is text to display at the start of the footnote.
% The \icmlEqualContribution command is standard text for equal contribution.
% Remove it (just {}) if you do not need this facility.

\printAffiliationsAndNotice{}  % leave blank if no need to mention equal contribution
%\printAffiliationsAndNotice{\icmlEqualContribution} % otherwise use the standard text.


%%%%%%%%%%%%%%%%%%%%%%%%%%%%%%%%%%%%%%%%%%%%%%%%%%%%%%%%%%%%%%%%%%%%%%%%%%%%%%%%%%%%%%%%%%%%%%%%
\begin{abstract}
Leave it blank for now.
\end{abstract}


%%%%%%%%%%%%%%%%%%%%%%%%%%%%%%%%%%%%%%%%%%%%%%%%%%%%%%%%%%%%%%%%%%%%%%%%%%%%%%%%%%%%%%%%%%%%%%%%
\section{Background}


(1)Probability Definition
$$p\left(s^{\prime}, r \mid s, a\right)=\operatorname{Pr}\left\{S_{t}=s^{\prime}, R_{t}=r \mid S_{t-1}=s, A_{t-1}=a\right\}$$
$$\sum_{s^{\prime} \in S} \sum_{r \in R} p\left(s^{\prime}, r \mid s, a\right)=1 \quad \forall s \in S, a \in A$$
$$p\left(s^{\prime} \mid s, a\right)=\sum_{r \in R} p\left(s^{\prime}, r \mid s, a\right)$$
$$r(s, a)=\sum_{s^{\prime} \in S} \sum_{r \in R} \left(r * p\left(s^{\prime}, r \mid s, a\right)\right)$$
$$r\left(s, a, s^{\prime}\right)=
\frac{\sum_{r \in R} \left(r * p\left(s^{\prime}, r \mid s, a\right)\right)}
{p\left(s^{\prime} \mid s, a\right)}$$


(2)Bellman Equations

v(s):
$$v_{\pi}(s)=E_{\pi} \left[G_{t} \mid S_{t}=s\right] \quad \forall s \in S$$
$$v_{\pi}(s)=E_{\pi} \left[R_{t+1}+\gamma G_{t+1} \mid S_{t}=s\right] \quad \forall s \in S$$
$$v_{\pi}(s)=\sum_{a \in A} \left(\pi(a \mid s) * q_{\pi}(s, a)\right) \quad \forall s \in S$$
$$v_{\pi}(s)=\sum_{a \in A} \pi(a \mid s) \sum_{s^{\prime}, r} p\left(s^{\prime}, r \mid s, a\right) * 
\left[r+\gamma E_{\pi} \left[G_{t+1} \mid S_{t+1}=s^{\prime}\right]\right] \quad \forall s \in S$$
$$v_{\pi}(s)=\sum_{a \in A} \pi(a \mid s) \sum_{s^{\prime}, r} p\left(s^{\prime}, r \mid s, a\right) * 
\left[r+\gamma v_{\pi}\left(s^{\prime}\right)\right] \quad \forall s \in S$$

q(s, a):
$$q_{\pi}(s, a)=E_{\pi} \left[G_{t} \mid S_{t}=s, A_{t}=a\right] \quad \forall s \in S, a \in A$$
$$q_{\pi}(s, a)=E_{\pi} \left[R_{t+1}+\gamma G_{t+1} \mid S_{t}=s, A_{t}=a\right] \quad \forall s \in S, a \in A$$
$$q_{\pi}(s, a)=\sum_{s^{\prime}, r} p\left(s^{\prime}, r \mid s, a\right) * 
\left[r+\gamma E_{\pi} \left[G_{t+1} \mid S_{t+1}=s^{\prime}\right]\right] \quad \forall s \in S, a \in A$$
$$q_{\pi}(s, a)=\sum_{s^{\prime}, r} p\left(s^{\prime}, r \mid s, a\right) * 
\left[r+\gamma v_{\pi}\left(s^{\prime}\right)\right] \quad \forall s \in S, a \in A$$
$$q_{\pi}(s, a)=\sum_{s^{\prime}, r} p\left(s^{\prime}, r \mid s, a\right) * 
\left[r+\gamma \sum_{a^{\prime} \in A} \left(\pi\left(a^{\prime} \mid s^{\prime}\right) * 
q_{\pi}\left(s^{\prime}, a^{\prime}\right)\right)\right] \quad \forall s \in S, a \in A$$

Optimal Equations:
$$v_{*}(s)=\max_{a \in A} q_{\pi *}(s, a) \quad \forall s \in S$$


(3)Dynamic Programming

Policy Improvement Theorem:
$$E_{\pi^{\prime}} \left[q_{\pi}\left(s, \pi^{\prime}(s)\right)\right] \geq v_{\pi}(s) \quad \forall s \in S$$
$$\pi^{\prime} \geq \pi \quad \leftarrow \rightarrow \quad v_{\pi^{\prime}}(s) \geq v_{\pi}(s) \quad \forall s \in S$$

Policy Evaluation:
$$v_{k+1}(s)={E}_{\pi} \left[R_{t+1}+\gamma v_{k}\left(S_{t+1}\right) \mid S_{t}=s\right]$$
$$=\sum_{a} \pi(a \mid s) \sum_{s^{\prime}, r} 
p\left(s^{\prime}, r \mid s, a\right)\left[r+\gamma v_{k}\left(s^{\prime}\right)\right]$$

Policy Improvement:
$$\pi^{\prime}(s)=\underset{a}{\arg \max} q_{\pi}(s, a)$$

Value Iteration:
$$v_{k+1}(s)=\max_{a} \mathbb{E} \left[R_{t+1}+\gamma v_{k}\left(S_{t+1}\right) \mid S_{t}=s, A_{t}=a\right]$$


%%%%%%%%%%%%%%%%%%%%%%%%%%%%%%%%%%%%%%%%%%%%%%%%%%%%%%%%%%%%%%%%%%%%%%%%%%%%%%%%%%%%%%%%%%%%%%%%
\section{Proposed Solution}

(1)Frozen Lake


%%%%%%%%%%%%%%%%%%%%%%%%%%%%%%%%%%%%%%%%%%%%%%%%%%%%%%%%%%%%%%%%%%%%%%%%%%%%%%%%%%%%%%%%%%%%%%%%
\section{Numerical Results}

Your experiment results should be here. You should add the figure/table if necessary.



\bibliography{example_paper}
\bibliographystyle{icml2023}



\end{document}


% This document was modified from the file originally made available by
% Pat Langley and Andrea Danyluk for ICML-2K. This version was created
% by Iain Murray in 2018, and modified by Alexandre Bouchard in
% 2019 and 2021 and by Csaba Szepesvari, Gang Niu and Sivan Sabato in 2022.
% Modified again in 2023 by Sivan Sabato and Jonathan Scarlett.
% Previous contributors include Dan Roy, Lise Getoor and Tobias
% Scheffer, which was slightly modified from the 2010 version by
% Thorsten Joachims & Johannes Fuernkranz, slightly modified from the
% 2009 version by Kiri Wagstaff and Sam Roweis's 2008 version, which is
% slightly modified from Prasad Tadepalli's 2007 version which is a
% lightly changed version of the previous year's version by Andrew
% Moore, which was in turn edited from those of Kristian Kersting and
% Codrina Lauth. Alex Smola contributed to the algorithmic style files.
